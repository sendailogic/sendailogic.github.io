%%%%%%%%%%%%%%%%%%%%%%%%%%%%%%%%%%%%%%%%%%%%%%%%%%%%%%%%%%%%%%%%%%%%%%%%%
% Config
%%%%%%%%%%%%%%%%%%%%%%%%%%%%%%%%%%%%%%%%%%%%%%%%%%%%%%%%%%%%%%%%%%%%%%%%%
\documentclass[a4paper]{article}
\usepackage[utf8]{inputenc}
\usepackage{amsthm,amsmath,amssymb,amsfonts}
\usepackage{xcolor}

% package for japanese text
\usepackage{CJKutf8}
% commands for japanese
\newcommand{\beginjapanese}{\begin{CJK}{UTF8}{ipxm}}
\newcommand{\finishjapanese}{\end{CJK}}

% Title page
\title{Instructions for Updating the Sendai Logic Homepage}
\author{Leonardo Vieira Pacheco Braga}

\begin{document}
\maketitle

sendailogic.com is currently generated using \texttt{hugo} (gohugo.io) and hosted using \texttt{github pages} (https://pages.github.com).
It can be accessed via sendailogic.com or sendailogic.github.io.
The github folder is at
\begin{quote}
    https://github.com/sendailogic/sendailogic.github.io.
\end{quote}

On MacOS, I recommend using \texttt{homebrew} (brew.sh) to install \texttt{hugo} and \texttt{git}.
On Windows, {\color{red} I have no idea how to do these things. Whoever is able to do it teach me.}
The commands use relative paths from the folder \texttt{sendailogic.github.io}.

Note: this document assumes basic knowledge about using the command line of your operating system.
Any tutorial should give you enough information in order to use it.

\section{Commands for \texttt{hugo}}
\texttt{hugo} is a static site generator.
In order to update the Sendai Logic website, one only needs to write some simple files and compile the site using \texttt{hugo}.

The contents of the site are in the folder \texttt{sendailogic-hugo-source/content}.
A new post (for a seminar or event) will be a file \texttt{yyyy-mm-dd.md} on the folder \texttt{sendailogic-hugo-source/content/post}.
The simplest way to generate a file is copying an existing file and modify it.
The standard template for a post is
\begin{verbatim}
    +++
    title = "yyyy.mm.dd. speaker"
    date = yyyy-mm-dd
    description = "presentation title"
    +++
    presentation title

    <!--more-->

    - Date/Time: Month dd, yyyy (Friday) / 16:00 - 17:00

    - Speaker: Speaker (University)

    - Venue: Rm 1201, Science Complex A, Tohoku Univ.

    - Abstract: Abstract
\end{verbatim}
Note: the \texttt{date} field is the date of the creation of the file.
The date in the \texttt{title} field is the day of the presentation.
If \texttt{date} is some date in the future, the seminar page {\em will not appear on the front page}.

To edit the site, we use
\begin{verbatim}
    hugo server -s sendailogic-hugo-source
\end{verbatim}
This command is used to test the site.
This will start a server and the site will be accessible from localhost:1313/ .
If a file is modified while the server is running, the changes will be visible from the above link.

To compile the site, use
\begin{verbatim}
    hugo -s sendailogic-hugo-source -d ./..
\end{verbatim}
After editing the file, we upload the changes using \texttt{git}.

\section{Commands for \texttt{git}}
See any introduction for \texttt{git} for a complete introduction.

To download the file for the first time use
\begin{verbatim}
    git pull "https://github.com/sendailogic/sendailogic.github.io.git"
\end{verbatim}
To update the site from the second time on, use
\begin{verbatim}
    git pull origin master
\end{verbatim}
To see the current changes on the files, use
\begin{verbatim}
    git status
\end{verbatim}
To add (or remove) new files to a commit use
\begin{verbatim}
    git add file.ext
\end{verbatim}
To add all files currently tracked that were modified use
\begin{verbatim}
    git add -u
\end{verbatim}
We use
\begin{verbatim}
    git commit -m "messege"
\end{verbatim}
bundle the changes in a commit.
In place of \texttt{messege} write a short and clear description of the change.
In case there are many modifications to be made it may be useful to separate the changes into multiple commits.
At last, use
\begin{verbatim}
    git push origin master
\end{verbatim}
to upload the changes to GitHub.

Note that to upload changes you need to be logged into GitHub.
For instruction see https://docs.github.com/en/github/using-git/caching-your-github-credentials-in-git.
If you are updating from a shared computer, please log out after updating the site.
Also, use your own account to update the site and not the Sendai Logic shared account.
\end{document}